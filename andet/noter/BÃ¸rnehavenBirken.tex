
Specialbørnehave med almindeligt børnehave liv.

De har børn med diagnoser:
- Autisme spektrumforstyrrelse.
- ADHD, opmærksomhed og koncentrationsvanskeligheder.
- Tourettes eller ander verbale og motoriske tics.
Alle børnene bruger tidslægningsskeamer

Autisme:
- Kommunikationsvanskeligheder
- Magnlende eller forsinket sprogudvikling
- Sprogforståelses vanskeligheder, impressivt/ekspressivt.
- problemer med socialt samspil, mest generelle ting er svært.
- Picto selecter igennem hillerørd komunikationscenter.
- Kunne være dejligt hurtigt at kunne ændre på deres dagsplan.
- ikke mange detaljer på et billede, kan være forstyrende.
- gentagelser er gode, for mange forskellige ting kan være træls. (fx, vaske hænder tre gange om dagen, med billede)
- Kan godt lide rutiner/gentagelser.
- børn danner mønstre, det er vigtigt at holde øje med disse mønstre, og sørge for at det ikke er nogle mønstre der er problematiske, så man kan bryde mønstret og hjælpe barnet. 
- daglige aktiviter kan have brug for guidelines, men nogle ting kan være mindre praktisk at digitalisere, såsom vaske hænder, tærer hænder osv osv. 



overvejelser:
Børn lærer at snakke gennem apps "mekanisk", er det bedre at få dem til at snakke, men mekanisk, eller ikke?
Det er nemmere for at den voksne at gå i guardian mode på barnets ipad, og ændre skema, end at gøre det fra deres egne guardian ipad.
Skal helst kunne tilpases, og være meget fleksibelt.
Fleksibilitet med farverne for dagene, måske kunne et preset være de etablerede farver.
Måske lave ugeskemaet sådan så der er forskellige mugligheder for stying af tid på dagen, pil, det forsvinder eller andet.
Arbejds skemaer er anvisninger til aktiviteter i huset, dem sker der ikke ændringer til, men hvis barnet lige har lavet en aktivitet, og deres eget skema siger de skal lave det igen, skal det gerne være fleksibelt til at vise at de ikke længere behøver at gøre det.
Det vigtigste er at programmet bliver stabilt, men ville også være rart hvis det virkede på iOS også.
Presets til uger kunne være dejligt, måske kopiere den gamle uge og lave ændringer i den.
en uge ad gangen er fint, eller for hele dagen. selvom det er fredag, kan skema for hele ugen stadig vises.













