\section{Group Organization}
In the previous year, each group were assigned an area of responsibility:
\begin{itemize}
    \item Front-end
    \item Back-end
    \item Server
    \item Scrum
    \item Product Owner
\end{itemize}

However, they found it difficult to distribute knowledge between groups and advised against this structure. Instead, they advised the use of full stack groups, where each group works with all parts of the product by implementing user stories. This is the structure chosen to organize groups in this semester. Each group had one specialized skill member for each of the topics: 

\begin{table}[htbp]
\begin{tabularx}{\textwidth}{l|X}
\hline
 Front-end  & Discuss and share knowledge about front-end of the GIRAF client application.  \\ \hline
 Back-end   & Discuss and share knowledge about the API, the communication with database, the clients communication with the API and the clients back-end.   \\ \hline
 Server     & Discuss and share knowledge about which servers are necessary, their functionality and the maintenance of them.\\ \hline
\end{tabularx}
\end{table}

The skill groups were an attempt to solve the problem of communication between development groups, while still allowing the development groups to discuss the development and maintenance of each of the three parts of the product.
Despite the change to skill groups for the different parts of the product, scrum and product owner responsibility was still assigned to two of the development groups. The scrum group had the responsibility of making decisions about and controlling the process, while the product owner group had the responsibility of being in contact with the customers to find requirements and make sure the product fits the customers needs.