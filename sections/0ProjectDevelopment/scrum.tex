\section{Scrum}
\label{sect:Scrum}
%Scrum (scrum of scrums)
The process used to structure the work between groups was based on the Scrum framework, but changed slightly along the way. More specifically, we used a scrum of scrums approach. This means that one group acted as the scrum master (the scrum group), with the job of administrating the process. 
Work was organized into four sprints, with the goal of having a new release ready at the end of each sprint. 

%Product Backlog
A \textbf{product backlog} was available at all times, managed by the PO-group, which contained prioritized user stories covering the work that needed to be done on the product.

%Sprint Planning
The first sprint started with a \textbf{sprint planning} meeting, where all group members of each group were present. The PO-group started by presenting the user stories and the sprint goal. The development groups then selected the user stories they wanted to work with and created an estimate in story points of how long it would take to complete it. When all groups had been assigned user stories and the PO-group had approved them, the meeting was over. 

However, this approach was not used in the following sprints. It was changed to a \textbf{sprint introduction} meeting, where the PO group still presented the sprint goal and high priority user stories, but user stories were not assigned during the meeting. Instead, development groups would assign themselves to user stories when they needed more work during the sprint. 

%Standup Meetings
During the sprints, a 15 min \textbf{standup meeting} was held every week, where one representative from each development group participated. In the meeting, the participants discussed the progress of their group along with any possible conflicts they may have had, or dependencies with the work of other groups.

%Sprint Review
A \textbf{sprint review} was held at the end of every sprint. The goal of this meeting was to evaluate the product. Each development group was required to have one representative present at this meeting. The representative would present the group's implementation and the state of their user story(ies). Based on how the user story satisfied the user's needs, the PO-group would either accept or reject the implementation.
Throughout the meeting the rest of the groups were encouraged to provide feedback.

%Sprint Retrospective
The \textbf{sprint retrospective} meeting was also held at the end of each sprint, and was used to evaluate and improve the multi-project process. Everyone was present at the meeting, where development groups were mixed together to discuss ideas and problems across groups. These were written down, and in the end everyone voted on them, to find the most important ones.

%\subsubsection{Scrum of Scrums}
%From recommendations from the last semester, the concept of scrum of scrums has been introduced in this semesters multi-project. Scrum of scrums means that the development process will have one master \todo{Andet ord} scrum containing an overall product backlog from which tasks will be assigned to development groups. These tasks can then be administrated by the development groups into a smaller scrum. The administrators of the master scrum \todo{Andet ord} are the scrum group.

More details about the process can be found in the process manual written by the scrum group, SW602F19 \citep{cite:projectManualSW602F19}.