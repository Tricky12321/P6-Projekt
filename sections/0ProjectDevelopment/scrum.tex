\section{Scrum}
\label{sect:scrum}
This section describes the general principles of the Scrum framework and is based on the Scrum guide written by \cite{cite:scrumGuide}.
Scrum is a framework used to address complex problems, while retaining a high level of productivity and creativity in order to deliver products of high value to the customers.

\subsubsection{Scrum Team}
The Scrum team features three different roles: the scrum master, the product owner and the development team. These roles cover different kinds of responsibilities. 
The scrum team should have a wide variety of competences, allowing them to solve issues in different parts of a product, thereby being disjoint from other groups.
Products in the scrum team are created incrementally, allowing the stakeholders to test the each version of the product and provide feedback along the way. 
%The scrum team is self organising, meaning that the team retains all the competencies needed to solve a task, and completes this task without being guided by anyone outside the scrum team.

%The scrum team delivers products incrementally and iteratively, in order to enhance customer feedback.

%Scrum Teams deliver products iteratively and incrementally, maximizing opportunities for feedback. 


\begin{itemize}
\item To uphold the principles of Scrum, a \textbf{scrum master} is assigned to enforce these by always being available for the development teams, and by always making sure that the rest of the team follows the guidelines. The scrum master also helps manage communication with the scrum team, to make sure they keep a high velocity.

\item The \textbf{product owner} helps the development team in delivering a product that brings the most value to the customer. The product owner is responsible for the product backlog. This includes ordering items of the backlog relative to what brings the highest value to the customer. The product owner serves as the voice of the stakeholders, and all changes to the product backlog must be through the product owner.
Furthermore, the product owner ensures that the product backlog is fully transparent for everyone on the scrum team. Additionally, the product owner needs to ensure that the team understands the items in the product backlog.

\item The responsibility of the \textbf{development team} is to develop the product through the increments created during the sprints. 
The size of the development team should be balanced. Too few developers means that the team does not have broad enough competences, and a too big team means that the agility of the team could be compromised.
The development team is self sufficient with a broad skill set and the team itself decides how they want to develop items from the backlog.
\end{itemize}

\subsubsection{Scrum Events}
Through the five different events in Scrum, it should be possible to keep the process agile without wasting time on unnecessary meetings. The following points will describe these five events. Scrum puts a lot of emphasis on keeping the events within a predetermined time frame. 

\begin{itemize}
\item The \textbf{sprint} is a period of development (no longer than four weeks), where the increment is developed. The events sprint planning, daily scrum, sprint review and sprint retrospective are all a part of the sprint. Typically, a sprint is neither cancelled, lengthened or shortened. In certain occasions this might be necessary, but it is preferably avoided.

\item The purpose of the \textbf{sprint planning} event is to plan which items from the product backlog should be on the upcoming sprint's sprint backlog. The workload for the items is based on earlier experiences from earlier sprints. The scrum master will make sure the meeting is held and that everyone understands what is going to happen. The development team can estimate the time a task will take by using a planning technique, eg. planning poker. 
Based on the outcome of planning poker, the development team estimates how many items they can handle and in this way approximate the amount of items on the backlog they will be able to complete.

\item The \textbf{daily scrum} is used as a daily briefing in the development team. Each member of the development team talks about what they did yesterday, what they will do today, and which obstacles they might face during the completion of their task. By doing this, the collaboration between the members of the development team is improved and the progress of the team becomes more transparent.

\item The \textbf{sprint review} is a short meeting (less than four hours) which is held in the last phase of a sprint, and includes the scrum team and the stakeholders. 
The meeting is used to validate the sprint backlog items and ensure that they properly fulfil the stakeholders' expectations. During this meeting, the product owner will inform the attendees which items from the backlog has been implemented.
The product backlog should be updated according to changed expectations or additional requirements from the stakeholders.

\item The \textbf{sprint retrospective} is used for the team to reflect on the process. The sprint retrospective is held after the sprint review and before sprint planning for the next sprint. 
The scrum team discusses pros and cons of the sprint, how and if any of these should be used in the following sprint, or if new ideas should be introduced.
\end{itemize}

\subsubsection{Scrum Artefacts}
Scrum artefacts are a way of representing the work of the project, and how the work is valued in form of importance and complexity. Scrum artefacts should allow for shared understanding of the project through the whole scrum team.

\begin{itemize}
\item The \textbf{product backlog} contains the work to be done for the product, in a prioritised order. The product backlog is solely managed by the product owner. As the product evolves, so does the product backlog, adding new features, improvements and fixes.

\item The \textbf{sprint backlog} is a list containing the work to be done in the current sprint. This work is selected from the product backlog by the developers, who estimate how much work they are able to finish in the sprint and what work is deemed necessary to meet the sprint goal.
The development team manages the sprint backlog from start to end. When work is finished or requirements change, the development team makes sure this is reflected in the sprint backlog. 
The sprint backlog is also used to monitor the sprint progress, so the development team knows whether they are behind or not, in achieving the sprint goals. 

%\item An \textbf{increment} is all the work accomplished in the completed sprint including all previous work done. Each increment is part of the final release. Every increment must be in functional order and fulfil the conditions for done set by the scrum team. 

\item An \textbf{increment} is a version of the product reached through the work done in a sprint. Each increment must be in a state that can be considered done and should strive towards the overall vision for the product in development. 

%https://www.scrumguides.org/docs/scrumguide/v2017/2017-Scrum-Guide-US.pdf#zoom=100
\end{itemize}

\section{Scrum in the GIRAF Project}
\label{sect:scrumInGiraf}
%Scrum (scrum of scrums)
The process used to structure the work between the different groups in the GIRAF project was based on the Scrum framework explained in section~\ref{sect:scrum}, but changed slightly along the way. More specifically, we used a Scrum of Scrums approach. This meant that a large group of people was split into smaller groups, which in our case was the people working on the GIRAF project, who were split into the development groups. The specific guidelines for the process of the GIRAF project can be found in the process manual created by the process group, which can be found on the GIRAF wiki page on GitHub \citep{cite:processManualSW602F19}.

\subsubsection{Scrum Team}
\begin{itemize}
\item Since we were split into development groups, the Scrum roles were assigned per group instead of for each person.
This meant that one group acted as the \textbf{scrum master} (the process group), with the job of administrating the process. They arranged all the meetings held during the process and ensured that the Scrum techniques were used correctly. 

\item Another group had the responsibility of the \textbf{product owner} (the PO group). They would take care of all customer interaction, as well as the creation and management of the product backlog. The management of the product backlog included prioritising the items, creating and deleting items if needed, in order to bring the most possible value to the customers. 

\item The rest of the development groups acted as \textbf{development team}, implementing the items from the product backlog. 
The process group and PO group were also part of the development team, in order to let everyone be a part of the product development. This goes against Scrum, since the product owner should not be part of the development team.
\end{itemize}

\subsubsection{Scrum Events}
\begin{itemize}
\item In this project, work was organised into four \textbf{sprints}, with the goal of having a new release ready at the end of each sprint. The first two sprints lasted three weeks, while the last two sprints lasted two weeks.

%Sprint Planning
\item The first sprint started with a \textbf{sprint planning} meeting, where all group members of each group were present. The PO group started by presenting the issues and the sprint goal. The development groups then selected the issues they wanted to work with and created an estimate, in story points, of how long it would take to complete them. When all groups had been assigned issues and the PO group had approved them, the meeting was over. 
However, this approach was not used in the following sprints. It was changed to a \textbf{sprint introduction} meeting, where the PO group still presented the sprint goal and high priority issues. Issues were not assigned during the meeting, but instead, the development groups would assign themselves to issues when they needed more work during the sprint. 
As our first two sprints were hectic, this made the estimation process of the sprint planning very difficult, and did not let us create a proper foundation for future estimations. As a result of this, the process group decided that it would be better to allow development groups to assign themselves to issues throughout the sprint if they had excess time.  

%Standup Meetings
\item During the sprints, daily scrum meetings were not held between development groups. Instead, a similar 15 minutes \textbf{standup meeting} was held 1-3 times a week, where one representative from each development group participated. At the meeting, the participants discussed the progress of their group's issues along with any possible conflicts they may have had, or dependencies with the work of other groups.
The reason we did not hold daily scrum meetings was that it was complicated to gather all groups every day. Not all groups had identical working hours, and one group worked at home. 1-3 meetings a week was the compromise used to gain some of the benefits of daily scrums. 

%Sprint Review
\item For the first two sprints, a \textbf{sprint review} was held at the end of the sprints. The goal of this meeting was to evaluate the product. Each development group was required to have one representative present at this meeting. The representative would present the group's implementation and the state of their issues. Throughout the meeting the rest of the groups were encouraged to provide feedback. Additionally, the PO group would use the sprint as a foundation for a user test, and the feedback from the user would then be used to refine elements of the application.
In the last two sprints, this was moved to a \textbf{release party}, held at the end of each sprint. Here, everyone would be present, and each group would demonstrate the features they finished throughout the sprint. This way, everyone could see what was achieved during the sprint and what the status of the application was.

%Sprint Retrospective
\item The \textbf{sprint retrospective} meetings held in the project followed the guidelines of Scrum. They were held at the end of each sprint and were used to evaluate and improve the multi-project process. Everyone was present at the meeting, where development groups were mixed together to discuss ideas and problems across groups. These were written down as proposals and gathered when no more ideas where found. During the first two sprints, evaluation of these proposals were done through voting to find the most important ones. 
This was changed in the last two sprints, since the voting method only allowed people to vote on their three favourite proposals. Instead, a questionnaire was released with all proposals and required everyone to choose whether they agreed, disagreed or were neutral with each proposal.
\end{itemize}


\subsubsection{Scrum Artefacts}
\begin{itemize}
%Product Backlog
\item A \textbf{product backlog} was available at all times, managed by the PO group, which contained prioritised issues covering the work that needed to be done on the product. Issues were added as new functionality or errors were discovered, either through development or through contact with the customers.

\item A \textbf{sprint backlog} was not used in this project. Instead of planning the work of the sprint in advance, issues were assigned along the way when groups needed more work. This meant that we did not estimate work for the entire sprint and set a sprint goal accordingly. This made the process of working on issues more flexible and easy for development groups, but arguably less structured and without a specific goal to work towards for each sprint. 

\item The \textbf{increments} are the releases we made for each sprint. To make sure each release was in a usable condition, each sprint ended with a release preparation period. This period lasted a couple of days and was used to finish remaining issues, review remaining pull requests, test the application and fix bugs. 
\end{itemize}

\subsubsection{Scrum Internally in the Group}
During development, we as a group only used a few principles from Scrum. We acted as part of the development team in the project, but did not assign different roles in our own group. For example, a product owner was unnecessary, since the PO group handled all contact with customers and management of the product backlog. 

We did not hold any official scrum meetings in the group either. The reason for this was that all work in the group was done together in the group room, so there was always a close internal communication in the group. We did, however, keep an overview over the issues we completed and reflected on problems in our process, but this was not done through official meetings. 
These can be seen in the "Our" parts of the sprint review and -retrospective sections of each sprint in the report. 
Due to the close communication in the group, we felt that holding official daily scrum meetings were not necessary to share knowledge and keep track of what people in the group were working on. However, this caused problems during the second sprint.

After sprint two, we did make use of a scrum board, to keep track of our work. This included both the issues we were assigned, as well as tasks for the report. The scrum board can be seen on figure~\ref{fig:scrumBoard}.
 \figur {H} %Placement
        {1} %Size
        {sections/0ProjectDevelopment/images/scrumBoard.jpg} %Filepath
        {Scrum board used in the project.} %Caption
        {fig:scrumBoard} %Label