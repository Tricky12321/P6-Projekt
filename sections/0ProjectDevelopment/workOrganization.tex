\section{Workflow}
\label{sect:workOrganization}
As mentioned in section~\ref{sect:versionControl}, the overall project workflow was structured using GitHub. The way it was done was by managing a product backlog on GitHub containing the work that needed to be done. The backlog was created by identifying this work, partly through what was received from previous year and partly through communication with customers. Both the creation and management of the backlog was done by the PO group. 

Everything in the product backlog was distributed into numbered \textbf{issues} of different sizes and priorities. Each repository (weekplanner, web-API, API-client and wiki) had their own set of issues, with their own numbering. This meant that both the weekplanner and web-API repositories could have an issue \#7. The vast majority of issues were on the weekplanner repository. Therefore, it should be assumed that issues stated in this report are from the weekplanner unless stated otherwise.
An issue would always consist of one of the following:

\begin{itemize}
    \item \textbf{User Story:} Description of a feature to implement, from the user's perspective. User stories follow the template:\\\newline
        \begin{minipage}{.5\linewidth}
        \textit{"As a <type of user>,\\}
        \textit{I would like <feature description>,\\}
        \textit{so that <reason for the feature>.".\\}
        \end{minipage}
    \item \textbf{Bug:} Something that needs to be fixed in the code, like an error or other unexpected behaviour.
    \item \textbf{Task:} Covers all other development tasks, which are neither user stories or bugs.
\end{itemize}

These issues would be assigned to the development groups, who would then have the responsibility of solving the issues. 
Each group should only have a few issues assigned at one time, to prevent groups from having issues they do not actively work on and which another group could have completed instead. 

When assigned to an issue, a group would create a new feature branch, as mentioned in section~\ref{sect:versionControl}, branching from the develop branch, to implement the issue in.
When an issue was complete, the group would create a pull request for the issue. This is a request to merge the feature branch into the develop branch, thus including it in the product. 
Before a pull request could be accepted and merged, two external reviews from other groups were necessary, along with a design approval from the PO group. This worked by having the process group assign two other groups for review. These groups would then look through the code, looking for errors, missing functionality and whether the code lives up to the coding standards. They would also compile the app and test functionality and run the tests to ensure that no other functionality was broken by the update.
After the reviewers accepted the pull request, it would be merged into the develop branch, the respective issue would be closed and the feature branch would be deleted. 
