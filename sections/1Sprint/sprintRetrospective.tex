\section{Sprint Retrospective}
\label{sect:RetrospectiveSprint1}
\subsubsection{Overall Sprint Retrospective}
\label{sect:TeamRetrospectiveSprint1}
The sprint retrospective was held as described in section~\ref{sect:Scrum}. The first part of the retrospective was to discuss the process, specifically what went bad and what went well in the process. This discussion was done in six groups, each with one developer from each developer group. 
If a group found a substantial problem that they wanted to change, they posted the proposal to a board on \url{dotstorming.com}. After all groups were done discussing, everyone participated in a voting on the proposals. 
Proposals with a substantial amount of votes were reviewed by the Scrum group, which then decided if they were problems worth changing and how they should be changed.
The table~\ref{table:retrospectiveIdeas} shows the top three ideas from the sprint retrospective.

\begin{table}[htbp]
\begin{tabularx}{\textwidth}{X|l}
\textit{Proposal}    & \textit{Votes}  \\ \hline
1. Each user story should also have a developer's point of view, with technical descriptions, so that each user story would include a developer's and a customer's point of view.  & 21  \\ \hline
2. Less meetings on days without lectures & 14 \\ \hline
3. Consider cancelling sprint planning & 11  \\ \hline
\end{tabularx}
\caption{The three most voted proposals.}
\label{table:retrospectiveIdeas}
\end{table}

Proposal 1: The idea with this proposal was to get a better description of the work to be done for each user story. This could help development groups when choosing user stories, as well as provide a starting point for the implementation of each user story. Even though this was the proposal with the most votes, it was not put into effect. 

Proposal 2: This concerns the problem that some groups preferred to work at home, so they were only present at the university for lectures. This meant that meetings planned on days without lectures required them to travel to the university just to attend one meeting, usually for only 15 minutes. To accommodate this, subsequent meetings were more carefully planned according to scheduled lectures. 

Proposal 3: After the sprint retrospective, cancellation of sprint planning was considered by the Scrum group, but they decided to change it instead. Subsequent sprint planning meetings functioned more as sprint introductions, rather than planning. In these meetings, user stories were not assigned and estimated among groups. Instead, groups picked user stories themselves throughout the sprint with approval from the PO group. What remained of the meetings was a presentation from the PO group, explaining the goals for the sprint and presenting the highest priority user stories for that sprint, as well as unfinished user stories from the previous sprint.


\subsubsection{Our Sprint Retrospective}
\label{sect:ourRetrospectiveSprint1}
As mentioned in the overall sprint goals in section~\ref{sect:1SprintGoals}, a lot of changes happened throughout the first sprint. First we had the offline support, then choose citizen in Xamarin, then the change to Flutter and lastly choose citizen and login in Flutter. This meant that we didn't have time to properly familiarize ourselves with a specific user story and follow a stable process. This makes it difficult to reflect on the process and what to do differently. The following shortly summarizes the process:

In the beginning of the development process with Xamarin, we attempted to program together as 5 man, to get an overview of the project. Shortly after, everyone in the GIRAF project team came to the agreement to change from Xamarin to Flutter. 
We subsequently continued to program as 5 man, but this time with the intention to learn to program in Flutter. 
After some time, we decided to program in pairs instead, to utilize our time more efficiently.

One thing to reflect on is the pair programming. This let us better utilize our work hours by distributing the work, while still having more people working together on a problem. This worked well for us, so we chose to use this in the following sprint as well.