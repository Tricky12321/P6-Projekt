\section{Sprint Retrospective}
\label{sect:retrospectiveSprint1}
In this section, proposals for changes to the next sprint is described. These proposals are found through the overall sprint retrospective and our retrospective.

\subsubsection{Overall Sprint Retrospective}
\label{sect:teamRetrospectiveSprint1}
The sprint retrospective was held as described in section~\ref{sect:scrum}. The first part of the retrospective was to discuss the process, specifically what went bad and what went well. This discussion was done in six groups, each with at least one member from each development group. 
If a group found a problem that they wanted to change, they posted the proposal to a board on \url{dotstorming.com}. 
After all groups were done discussing, everyone participated in a voting on the proposals. 
Proposals with a substantial amount of votes were reviewed by the process group, which then decided if the problems were worth changing and how they should be changed.
The table~\ref{table:retrospectiveIdeas} shows the top three proposals from the sprint retrospective.

\begin{table}[htbp]
\begin{tabularx}{\textwidth}{X|l}
\textit{Proposal}    & \textit{Votes}  \\ \hline
1. Each issue should also have a developer's point of view, with technical descriptions, so that each issue would include a developer's and a customer's point of view.  & 21  \\ \hline
2. Less meetings on days without lectures & 14 \\ \hline
3. Consider cancelling sprint planning & 11  \\ \hline
\end{tabularx}
\caption{The three most voted proposals.}
\label{table:retrospectiveIdeas}
\end{table}

\begin{itemize}
\item \textbf{Proposal 1:} The idea with this proposal was to get a better description of the work to be done for each issue. This could help development groups when choosing issues, as well as provide a starting point for the implementation of each issue. Even though this was the proposal with the most votes, the process group decided not to enforce it. 

\item \textbf{Proposal 2:} This concerns the problem that some groups preferred to work at home and were only present at the university for lectures. This meant that meetings planned on days without lectures required them to travel to the university just to attend one meeting, usually for only 15 minutes. To accommodate this, subsequent meetings were more carefully planned according to scheduled lectures. 

\item \textbf{Proposal 3:} After the sprint retrospective, cancellation of sprint planning was considered by the process group, but they decided to change it instead. Subsequent sprint planning meetings functioned more as sprint introductions, rather than planning. In these meetings, issues were not assigned and estimated among groups. Instead, groups picked issues themselves throughout the sprint with approval from the PO group. What remained of the meetings was a presentation from the PO group, explaining the goals for the sprint and presenting the highest priority issues for that sprint, as well as unfinished issues from the previous sprint.
\end{itemize}

\subsubsection{Our Sprint Retrospective}
\label{sect:ourRetrospectiveSprint1}
As mentioned in the overall sprint goals in section~\ref{sect:1SprintGoals}, a lot of changes happened throughout the first sprint. First we had the offline support issue, then citizen selection in Xamarin, then the change to Flutter and lastly citizen selection and login in Flutter. This meant that we did not have time to properly familiarise ourselves with a specific issue and follow a stable process. This made it difficult to reflect on the process and what to do differently. The following shortly summarises the process:

In the beginning of the development process with Xamarin, we all attempted to program together, to get an overview of the project. Shortly after, the GIRAF team came to the agreement to change from Xamarin to Flutter. 
We subsequently continued to program as five, but this time with the intention to learn to program in Flutter.
After some time, we decided to program in pairs instead, to utilise our time more efficiently.

One thing to reflect on is the pair programming. This let us better utilise our work hours by distributing the work, while still having more people working together on a problem. This worked well for us, so we chose to use this in the following sprint as well.