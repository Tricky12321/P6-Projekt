\section{Sprint Goals}
\label{sect:sprintGoals2}
The goals described in this section are partly about finishing the leftovers of unfinished issues from the previous sprint. As these issues were finished, we were able to pick up new issues.

\subsubsection{Overall Sprint Goals}
Since the first sprint started with rewriting the entire application in Flutter, sprint two focused on getting a release of the core functionality of the weekplanner application. This meant that the overall goal of the 2nd sprint was to be close to the level of functionality the Xamarin application had before the transition to Flutter. The goals for development where functionality, stability and usability. 

\subsubsection{Our Sprint Goals}
In the second sprint, a lot of time was used to finish and refactor the choose citizen issue from sprint 1. On top of this we worked with the following issues:

\begin{itemize}
    \item Issue \#83: Task "Load configuration from configuration file."
    \item Issue \#73: Task "Making the process of logging in as a developer less tedious, with a feature to fill in login information automatically."
    \item Issue \#7 (web-API): Task "API should time out after a while to prevent requests from crashing the API."
    \item Issue \#47: User story "As a guardian I would like to be able to choose a citizen so that I can choose who I’m editing the week plan for."
    \subitem Continuing the integration of the choose citizen screen.
    \item Issue \#48: User story "As a guardian I would like to be able to log into the system using a username and password, so that I can see my associated citizens."
    \subitem Continuing the development of the loading spinner.
    \item Issue \#88: Task "Create a notification widget to avoid duplicated code."
\end{itemize}