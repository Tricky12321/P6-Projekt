\section{Sprint Retrospective}
\label{sect:retrospectiveSprint2}
\subsubsection{Overall Retrospective}
\label{sect:teamRetrospectiveSprint2}
The sprint retrospective of sprint 2 followed the same procedure as the one from sprint 1, described in section~\ref{sect:teamRetrospectiveSprint1}, except that the discussions were done in four groups rather than six groups.
The table~\ref{table:retrospectiveIdeasSprint2} shows the top three most voted proposals from the sprint retrospective meeting.

%PR var langsomt i starten, blev bedre senere. Godt at sætte sig ind til gruppen

\begin{table}[H]
\begin{tabularx}{\textwidth}{X|l}
\textit{Proposal}    & \textit{Votes}  \\ \hline
1. The release preparation should be one day where all developers are gathered, so merging the features and preparing them for release can be done faster.  & 12  \\ \hline
2. When creating a pull request, always include a screenshot of the implemented screens & 9 \\ \hline
3. Linter gives a warning when the document has undocumented public members, consider making this an error, to force documenting the code and provide an easier handover. & 8  \\ \hline
\end{tabularx}
\caption{The three most voted proposals.}
\label{table:retrospectiveIdeasSprint2}
\end{table}

\textbf{Proposal 1:} The purpose of this proposal was to have a more efficient release preparation. The first release preparation was rather hectic, due to groups running back and forth between other groups or communicating over Slack to resolve occurring problems. This was rather time consuming and unstructured, so if instead the entire team was available at the same location, the process of communicating and discussing between groups would become easier, resulting in a faster and better release.

\textbf{Proposal 2:} This proposal was for other developers to have an easier job reviewing pull requests. If a pull request had a vague description, the reviewer might have a difficult job understanding exactly what they were reviewing. If the pull request included a picture of what had been created/changed, the reviewer could more easily understand what work had been done. Everyone seemed to agree, and pull requests from this point forward contained pictures of the work when possible.

\textbf{Proposal 3:} This concerned the problem that, at the time, there was a lot of undocumented code, even though we were not far into the project. Not only could the warnings be annoying, but not fixing them by documenting the code could lead to problems, both in future sprints and for future students taking over the development of the project. Making Linter see the warnings as errors instead would assure that the developers documented their work, avoiding undocumented code due to either laziness or forgetfulness.
The proposal was put in use later, but not in the way described. Instead of Linter seeing undocumented code as errors, it still considered them as warnings. However, the Azure Pipelines was changed to not let any pull requests go through if the code had any warnings.

\subsubsection{Our Retrospective}
\label{sect:ourRetrospectiveSprint2}
The second sprint was much more organized than the first sprint. The core of the project had been properly implemented in Flutter, every group had had time to familiarize themselves with Flutter, and plenty of issues were ready to be worked on. This let us properly use the system of issues, pull requests and reviews, as described in section~\ref{sect:workOrganization}. This created a good, structured workflow both between development groups and internally within groups. With this, we could assign our group to a couple of issues and then distribute the work within the group.

However, we also encountered problems during the sprint. Due to inadequate communication, we ended up having too many issues assigned. On top of that, we worked on several issues on the same branch, which made the implementation messy. This resulted in a big pull request that was difficult to review, and which lasted for over two weeks with over 100 comments. Preferably, one issue should be developed in one branch and result in one pull request. This was something we had to improve on in the following sprints. The plans we had for solving these problems can be seen in table~\ref{table:ourRetrospectiveIdeasSprint2}.

\begin{table}[ht]
\begin{tabularx}{0.9\textwidth}{X|X}
\textit{Problems from sprint.} & \textit{Planned solutions for following sprints.} \\\hline
Several issues in a single branch. &  More pair programming, to prevent that one developer is the only one that knows what a branch contains. \\ \hline
Not being transparent of what issues the group has acquired. & More transparency of what issues the development group is currently working on, could be handled by making use of a scrum board. \\\hline
Big pull requests that take a long time to solve. & Peer reviewing code internally in the group before a pull request is opened. \\\hline
\end{tabularx}
\caption{Thoughts about the sprint.}
\label{table:ourRetrospectiveIdeasSprint2}
\end{table}

%Better communication in the group