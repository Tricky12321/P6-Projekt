\section{About GIRAF}
% GIRAF is an Aalborg University project, which is worked on by multiple groups on the software 6 semester. GIRAF is a project with intention to develop tablet applications which can help autistic children in everyday tasks. GIRAF is aimed at autistic children with little to no verbal communication, here GIRAF is intended to provide tools to help with everyday use. There is currently only one functional app, which is the Weekplaner. The Weekplaner is an app with pictograms to formulate a schedule with pictures, since words and text is complicated and can frustrate the citizens. 

The GIRAF project is a multi-project for 6th semester software students at Aalborg University. 
The GIRAF project was started in 2011, coordinated by Ulrik Mathias Nymann, and is intended to function as a bachelor project for the 6th semester software students. 
The objective of the GIRAF project is for the software students to receive experience in working in multi-projects, preparing students to working on bigger projects in future companies.

%The project is passed on each year, to the new 6th semester students and is introduced by Ulrik, who also helps to introduce the students to customers for continuous communication, and help students to get an overview of the project and earlier development.

The product of the GIRAF project is a piece of software, developed to aid autistic people with communication complications. 
The purpose of the GIRAF project is to develop software that can provide scheduling assistance and various games and tools for educational purposes.
At the moment, the only functional application is a tool for scheduling, which is the weekplanner. 

The GIRAF applications are developed in cooporation with the customers: Børnehaven Birken, Egebakken and Enterne, which is a kindergarten, school and home for disabled respectively. 

\subsection{Weekplanner}
As people with autism often need a lot of structure in their everyday life, many institutions often use a weekplanner for this structuring. The weekplanner will often be a physical poster with pictograms showing which activity the user should do next. A physical weekplanner can be seen on figure~\ref{fig:physicalWeekplanner}. The Weekplanner in the GIRAF project tries to mimic this electronically.
Here, each column represents a day, where one or more activities can be placed on the day's schedule. 

\figur  {H} %Placement
        {0.7} %Size
        {sections/0Introduction/images/Visual-supports-3-728x440.jpg} %Filepath
        {Physical weekplanner for people in need of structure \citep{cite:PhysicalWeeknaplnner}.} %Caption
        {fig:physicalWeekplanner} %Label

%The goal of the GIRAF project is to deliver a final piece of software to the customers, but as of now, the state of GIRAF is too unstable to be in any use. 

%Could talk about what customers we have in this project

% This section describes the current state of the GIRAF project, and the state of the current GIRAF application build. The GIRAF project was last developed on in 2018, by the previous sixth semester software students. 

% \subsubsection{Status of the Project}
% Last semester the groups focused on re-writing the application to make use of an API, instead of directly interacting with the database through the application. 
% This resulted in the applications being incompatible for communication with the database, and the applications became unusable. The new focus was to make the applications usable again and running with the new back-end. 

% The product from the previously development is the Weekplanner application that now works with the new back-end, but the remaining applications remain incompatible.
% The back-end consists of an API, that is generated using Swagger.  

% Swagger is an API designer tool, that can be used to create and maintain an API. Swagger can generate client- and server-side code, which then can be implemented to perform the desired features. The API is a better way to interact with the database, because all database interaction can be handled on the server side, and updated without the need to update all clients. 

% The servers are structured using Kubernetes, which features several nodes that host different functionality: the Web-API, the database and a build server.

% The Weekplaner app only runs on Android, this has created some problems since some of the institutions only provide iPads to their citizens. Because of this, institutions expressed that an iOS version would be appreciated. The Weekplanner application is made with the Xamarin framework, which also compiles and runs on iOS. In the current state, some iOS dependencies are missing in order for the build to work.
% Releasing to both iOS and Android would allow more institutions to make use of the application.

% %hurtigt hvad skete der sidst, hvilke problemer gav det, og hvordan påvirkede det status på giraf
% % uddybende om swagger server etc.

% \subsubsection{Status of the Build}
% The Giraf Weekplanner application runs, but has major stability problems and is often crashing in the hands of the user. Since the target audience of the application is citizens with autism, having an unstable app can be frustrating and cause disturbance in the everyday life of the user.
% In addition to this the aesthetics of the application is also not up to date and seems unfinished, and a refinement of the design would be appreciated.