\chapter{Conclusion}\label{ch:conclusion}
%konkluder på de projekt goals vi satte i starten af rapporten
%Evaluate on how it was to work in a multiproject, and how much we learned
At the start of the multi-project, the PO group defined some goals for this year's multi-project, which can be seen in \ref{sect:projectGoals}. The two major things we wanted to improve this year was stability and usability.

The stability of the application was improved through multiple code reviews, both internal and external, and monkey testing by other groups. The release preparation period at the end of each sprint also made sure that each new issue in the sprint was thoroughly tested again before adding it to the release. By doing this, a lot of bugs were caught before each release was finished. 

The usability of the application was improved through usability testing with the customers, allowing the PO group to manage the product backlog in a way such that the most important features, according to the customers, had the highest priority.
This also helped with the creation of prototypes and the design guide. Following these made it easy to create a uniform design and layout, in the way the customers wanted it.
The statement from Børnehaven Birken, that they found the application to be intuitive, is a good indicator that the application has reached a high level of usability.

%simple to use
%Birken statement: "Intuitive and functional"

To conclude on the last goal, four additional features were added, which were not present in last year's version: Add pictogram from gallery, dragging pictograms, copying activities between days and timers on activities were added to the weekplanner.

Based on the customers' feedback stated in \ref{sect:customerFeedback} and the goals for this multi-project stated in \ref{sect:projectGoals}, we can conclude that we adequately fulfilled the goals set in the beginning of the multi-project.

Beside the product goals, this project was also, as described in section~\ref{sect:aboutGiraf}, about getting experience with working with multi-projects. As described in section~\ref{sect:reflection} we gained many new competences that can be useful in our future as software engineers when working on complex software projects in larger teams, thereby fulfilling this goal. 