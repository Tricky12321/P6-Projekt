\chapter{Recommendations for Future Projects}
\subsubsection{Do Not Rewrite Everything}
One of the biggest factors in restricting progress on the GIRAF project is that it has been rewritten several times. We did this as well, when we rewrote the entire front-end from Xamarin to Flutter. We advice that you do not rewrite it again, and instead focus on improving and adding to the current product. We feel that we have created a good framework in Flutter with good documentation in the wiki, and hope you will take over from here.

\subsubsection{Colocation}
During the project, almost every group worked in the group rooms every day. This was a big advantage, as it made it easy to simply go sit down with people to work on a problem or to solve a conflict or dependency. We recommend doing this, and warn that if some groups choose to work at home, it can complicate communication and slow down progress.

\subsubsection{Full Stack Groups}
We recommend using full stack groups, where all groups work on all parts of the product through user stories. This way, every group works on the same things, while still being able to efficiently distribute work. This also made all groups feel more like one working unit and made communication between groups more natural. 

\subsubsection{Skill Groups}
Using skill groups did not give us any major benefit. However, it might still be a good idea to use it, if done properly. An idea that, when choosing who should be in the skill groups, you should choose someone with knowledge in the area and/or who is motivated to work in the area. This helps prevent that a skill group has several people who do not know much about the area.
Through our experience, the server group is the most important group to have skilled dedicated people in, as it can be quite complicated. 

\subsubsection{Development Tasks as Issues}
We highly encourage you to take over and follow the structure we have created with using issues on GitHub as a product backlog, as described in section~\ref{sect:workOrganization}.
You should have a dedicated PO group, who manages the backlog and makes decisions on what is important and how the UI should look, based on communication with the customers.
The issue structure gives a good workflow and an easy way to distribute work and add new issues when new functionality is wanted or bugs are found.

\subsubsection{Communication}
Communication between groups is key for the project to be successful. We recommend using slack for general communication, or something similar. 
Another part of communication, which we found very useful, was stand-up meetings (see section~\ref{sect:communication}). It helped synchronise information between groups. Everyone always knew what everyone else was working on, which made it easier to know who to ask for certain problems, and it gave everyone an overview of the project status as a whole. We highly recommend holding one or two of these meetings a week. 

\subsubsection{Process}
Our recommendations when deciding what process to use, is that all of you read the process manual \citep{cite:processManualSW602F19}, and decide what would work for you. We did as previously stated and followed it more loosely because of the hectic start. However, we would prefer to have tried to follow it more strictly, but due to the circumstances, following it the way we did made more sense.
We would recommend you to follow it more strictly than we did.
Another recommendation would be that you have a dedicated process group, since holding meetings with all people involved in GIRAF will be very time consuming and not very efficient.

%%Skriv noget om, at der burde følges en mere striks process.

%More transparency about what is going on with the server. We had no idea, which caused a few problems. Set up some communication standard to share information of server work with development groups.