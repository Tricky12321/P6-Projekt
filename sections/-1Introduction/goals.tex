\section{Project Goals} % Casper
\label{sect:projectGoals}
Early in this semester, the PO group arranged meetings with the customers of the GIRAF project. All parties expressed that a stable release of the system should be highly prioritised. Therefore, the early sprints of the project focused on finishing a release that was stable and fit for use.

To ensure, that the product would progress towards a stable release, groups of the GIRAF project all agreed to fix any small bugs encountered during development. The definition of a small bug in this case was if the time needed to fix the bug was estimated to $\sim$5 minutes or lower. If the bug was not fixable within this period, the groups were encouraged to create an issue on the GitHub page (this process is explained further in section~\ref{sect:workOrganization}), and these bugs would then be fixed in the next sprint or by groups with excess time in their sprint.

In addition to this, the customers emphasised that the product of the GIRAF project should have a high level of usability. The reason for this was that they expressed concerns that their less IT familiar employees would have a hard time manoeuvring and using a complex application, and as a consequence might end up not using it at all.

So in summary, the primary goals of the early sprints were to have a stable release, with a high level of usability.

In the later sprints, after a stable core had been implemented, the focus changed more towards adding new functionality. 
The goal was then to make the application more useful by adding features that would benefit the customer the most. Which features to implement was chosen based on communication with the customers.

While focusing on implementing new functionality, it was also a goal to keep the application stable, preventing bugs and crashes. At the same time, usability was also kept as a goal throughout the sprints, making sure that the design was user friendly and fitted the needs of the customers.
