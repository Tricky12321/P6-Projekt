\section{About GIRAF}
\label{sect:aboutGiraf}
The GIRAF project is a multi-project for 6th semester software students at Aalborg University. 
The GIRAF project was started in 2011, coordinated by Ulrik Mathias Nymann, and is intended to function as a bachelor project for the 6th semester software students. 
The objective of the GIRAF project is for the software students to receive experience in working in multi-projects, preparing them to work on bigger projects with more people in the future.

The purpose of the GIRAF project is to develop software for autistic people, which can provide scheduling assistance and various games and tools for educational purposes.
At the time of beginning this project, the only functional application was a tool for scheduling, which was the weekplanner. 

The GIRAF applications are developed in cooporation with the customers:
\begin{itemize}
    \item Børnehaven Birken, a kindergarten.
    \item Egebakken, a school.
    \item Enterne, a home for disabled.
\end{itemize}

\subsection{Weekplanner}
As people with autism often need a lot of structure in their daily life, many institutions use a weekplanner for this. A weekplanner will often be a physical poster with pictograms showing which activity the citizen should do next. A physical weekplanner can be seen on figure~\ref{fig:physicalWeekplanner}. The weekplanner of the GIRAF project tries to mimic this electronically. Here, each column represents a day, where activities can be placed on the day's schedule. 

\figur  {H} %Placement
        {0.7} %Size
        {sections/-1Introduction/images/visual-supports-3-728x440.jpg} %Filepath
        {Physical weekplanner \citep{cite:physicalWeekplanner}.} %Caption
        {fig:physicalWeekplanner} %Label
