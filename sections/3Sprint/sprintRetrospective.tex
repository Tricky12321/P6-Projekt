\section{Sprint Retrospective}

\subsubsection{Overall Sprint Retrospective}
The sprint 3 retrospective differed from the previous sprints, in that a voting system was no longer used to find the best proposals. Instead, a questionnaire was created with all proposals, with the options to agree, disagree or be neutral about each proposal.
This way, everyone gave their opinion on all proposals, instead of only three. Because of this, several proposals were accepted and used moving forward. These proposals, chosen by the process group, were as follows:

\begin{itemize}
    \item During standup meetings, group representatives should explain more about their issues and how they plan to complete them, to make it easier to find dependencies between the issues of different groups.
    \item If assigned to an issue, which is blocked by another issue, the group should note in the issue which one blocks it and then unassign themselves from the issue. This way, it does not look like they are currently working on it.
    \item There should be one pull request for one issue, unless two issues overlap.
    \item When reviewing a pull request, one should look through the entire pull request, instead of finding a few errors and then waiting until it is fixed to review the rest.
    \item When assigned to review a pull request, the group should attempt to review it the same day. If this is not possible, the group should contact the owner of the pull request. 
    \item Be more active on Slack to increase communication, check at least once a day.
    \item Contact the PO group if someone is in need of more work.
    \item If a group does not think they can finish a given issue within the current sprint, they should let the PO group know about it. 
\end{itemize}

\subsubsection{Our Sprint Retrospective}
Overall, the process for sprint 3 went very well. We followed the positive results from sprint 2, benefiting from the workflow system we had established. This meant that we could easily find work and distribute it in the group, resulting in a productive sprint with several completed issues. Additionally, we managed to complete all assigned issues in the sprint, having no unfinished work.

We also solved the problems from sprint 2, where miscommunication caused problems with too many issues, several issues in the same branch and a pull request which was too long. 
We used the solutions planned in the sprint retrospective of sprint 2, as shown in table~\ref{table:ourRetrospectiveIdeasSprint2} in section~\ref{sect:ourRetrospectiveSprint2}. 
We did more pair programming to distribute knowledge, which helped manage the work of each issue. 
We started using a scrum board, which helped give an overview of what work had to be done, both our assigned issues, work on the report and other tasks. 
We also introduced the requirement that all completed issues had to be reviewed internally by another group member before a pull request could be opened. This allowed us to find some errors early, which resulted in our pull requests being accepted more quickly in external reviews. 

%Any bad parts?