\section{Sprint Retrospective}

\subsubsection{Overall Sprint Retrospective}
Even though sprint 4 was the last sprint, a retrospective meeting was still held to reflect on the process. Like in sprint 3, a questionnaire was created with all the proposals from the retrospective, so people could vote whether they agreed or not. Since no further sprints were held, the proposals were not put in use, but they might be useful for next year's students. The proposals with common agreement, chosen by the process group, were the following:
\begin{itemize}
    \item Some reviews of pull requests were used to enforce personal preferences for the code. This should not be the case. Reviews should only be used to find errors, bad practices and breaking of rules decided on in the project.
    \item The server felt isolated from the rest of the project, with only a few people managing it. It was difficult to know what was going on with the server. Communication should be improved through Slack and skill group meetings.
    \item Not really a proposal, but there was a general agreement that reviews were being done much faster, which was a great improvement. 
    \item Communication would be easier if all groups were present in the group rooms. In this project, one group worked at home every day.
    \item Everyone should be available on Slack from 9-15 during all work days, and respond within a reasonable amount of time.
    \item There should be a specific naming convention for branches, to give a better overview.
\end{itemize}
%Indsæt itemize
%Andre specielle ting siden det nu var sidste sprint?

\subsubsection{Our Sprint Retrospective}
Sprint 4 did not go as well as sprint 3. The process itself was fine and followed the success of the previous sprint. However, the issues we were assigned, concerning the timers, took a long time to complete. On top of this, we were waiting for another group to finish an issue regarding an endpoint in the api, which was needed for our solution. We could have used another endpoint, but it would not be as efficient, so we chose to wait for the new endpoint. This meant that we were under a lot of time pressure during the last couple of days and had to work overtime to get the issue completed in time. 

Arguably, the problems were not due to the process, but rather due to the size and complexity of the issues and the choice of waiting for the endpoint. The reason we were assigned these issues were because they were considered important, and based on our previous performance, the PO group assessed that we would be able to get it done in time. 

