\chapter*{Readers Guide}
\addcontentsline{toc}{chapter}{Readers Guide}
This report is structured into three parts: Introduction, Sprints and Evaluation. 
The Introduction part introduces the project, stating the overall goals for the project and which technologies and tools were used.
It also describes the project development process, how work in the multi-project setting was organised.

The Sprints part has four chapters, each containing a sprint in the project. Each sprint starts with a Sprint Goals section, including both the sprint goals for the entire GIRAF project, as well as our own goals in the group. After this, all work done in the sprint is documented. In the end of each sprint, a sprint review and sprint retrospective is described, also both for the GIRAF project and for ourselves.

The Evaluation part includes our reflection on the project and our conclusion, as well as a chapter with our recommendations for the students who will continue work on the GIRAF project the following year.\\

Throughout the report we will talk about issues, described in section~\ref{sect:workOrganization}. Issue numbering follows the GitHub provided issue numbers. If not otherwise explicitly stated, assume that issues are numbered from the weekplanner repository.
Examples of issues:
\begin{itemize}
    \item Issue \#9: Making the Application Available Offline.
    \subitem This is a weekplanner issue.
    \item Issue \#11 (web-API): Back-end returns \textasciitilde 1 GB data on every pictogram search.
    \subitem This is a web-API issue.
\end{itemize}

Citations in this report are following the Harvard style. All the sources used can be found in the bibliography at the end of the report. Here, you can find the author of the source, followed by the name of the source, an ISBN number or an URL, the publication year, and which date the source was last visited.
If a citation is used after a period it should be regarded as a source for the entire paragraph, whereas if the citation is used before a period, it is the source for the sentence in which it is stated. 

%Sources in the report are in Harvard-style. In report sections the source states the first author and the year for release. In the bibliography, all authors are stated, together with the name of the used source, the publishing year, optionally the URL for the source or the ISBN number of the book and the date for when the source was visited. If a source is used for a whole section, the source is placed after the last period. If the source is before the period, the source is used for a part of the section.


\section*{Keywords}
\textbf{Development Groups} are the groups created at the start of the project period, in which the development of this project is done.\\\\
\textbf{Skill Groups} are groups formed from skill members from each development group. There are three skill groups, each with responsibility for a part of the product: front end, back end or server. The skill groups decide the standard for their part of the product and how to maintain it.\\\\
\textbf{Skill Member} is a member of one of the three skill groups. There is one skill member from each development group, for each skill group. It is the skill members' responsibility to communicate knowledge from the skill groups to their respective development groups, and make sure their group follow the established standards.\\\\
\textbf{PO} Product Owner.\\\\
\textbf{UI} User Interface.\\\\
\textbf{NFS} Network File System.\\\\
\textbf{Citizens} are autistic people from an autistic daycare facility.\\\\
\textbf{Guardians} is defined as employees from an autistic daycare facility.\\\\
\textbf{Testing} is used throughout the report. When used, it should be regarded as either unit test or integration test. Which one of these depends on the context. See section~\ref{sect:testing} for more information.
